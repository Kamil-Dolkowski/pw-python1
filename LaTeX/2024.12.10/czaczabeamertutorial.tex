\documentclass[10pt,xcolor={dvipsnames}]{beamer}



%% styl prezentcaji
\usetheme{Berlin}
% inne  style
% metropolis, AnnArbor, Antibes, Bergen, Berkeley, Berlin, Boadilla, boxes, CambridgeUS, Warsaw, Copenhagen, Darmstadt, default, Dresden, Frankfurt, Goettingen, Hannover, Ilmenau, JuanLesPins, Luebeck, Madrid, Maloe, Marburg, Montpellier, PaloAlto, Pittsburg, Rochester, Singapore, Szeged, Warsaw, classic

%% schemat kolorów
\usecolortheme[named=Brown]{structure}
% inne schematy
% albatross, beaver, beetle, crane, default, dolphin, dov, fly, lily, 
% orchid, rose, seagull, seahorse, sidebartab, structure, whale, wolverine

%% schematy fontów 
\usefonttheme{default}
% default, professionalfonts, serif, structurebold, 
% structureitalicserif, structuresmallcapsserif

\usepackage{amsmath}
\usepackage{amssymb}
\usepackage{latexsym}

\usepackage{polski}
\usepackage[utf8]{inputenc}

\usepackage{url}

\title{Prezentacje w pakiecie Beamer}
\author{Cezary Obczyński}
\date{\today}

\usepackage{array}
\usepackage{graphicx}
\usepackage{xcolor}
\usepackage{pgf} 
\usepackage{overpic}
\usepackage{wrapfig}
\usepackage{multimedia}
\usepackage{hyperref}
\usepackage{listings}
\usepackage{tabularx}   
\usepackage{booktabs}
\usepackage{multicol}


%\usefonttheme[onlymath]{serif}
\beamertemplatefootpagenumber % w stopce pojawia się numer slajdu 
%\setbeamertemplate{bibliography item}[text]
%\setbeamertemplate{navigation symbols}{}
%\setbeamertemplate{theorems}[numbered]
%\setbeamertemplate{theorems}[ams style]

\newtheorem{twierdzenie}{Twierdzenie}
\newtheorem{definicja}{Definicja}

%\theoremstyle{example}
\newtheorem{przyklad}{Przykład}


\begin{document}
\maketitle

\begin{frame}
	\frametitle{Spis treści}
	\tableofcontents
\end{frame}


\section{Liczby}

\begin{frame}
\frametitle{Pierwszy slajd -- liczba $e$}
Liczba $e$ pojawiła się w matematyce  dopiero w XVI wieku za sprawą szkockiego matematyka Johna Napiera (Nepera), który ułożył tablice logarytmów, bardzo pomocne przy skomplikowanych obliczeniach astronomicznych. Logarytmy bowiem wymyślono, aby zamienić mnożenie na dodawanie. Przez setki lat, cudowna własność logarytmów, dzięki której z pomocą tablic lub dwóch linijek z~logarytmiczną skalą -- można było dodawać zamiast mnożyć, ułatwiała astronomom życie. 

\bigskip
Liczbę $e$ definiujemy jako granicę
\begin{equation*}
e = \lim_{n\to \infty} \left(1+\frac{1}{n}\right)^n
\end{equation*}
\end{frame}


\begin{frame}
\frametitle{Polecenie \texttt{pause} -- własności liczby $e$}
\begin{itemize}
\item 
Liczba $e$ jest niewymierna.
\pause
\item 
Liczba $e$ jest liczbą przestępną, tzn. nie jest pierwiastkiem żadnego wielomianu o współczynnikach całkowitych.
\pause
\item $e \approx 2.718281828459045235$.
\pause
\item Dla każdego $x\in\mathbb{R}$ zachodzi nierówność $e^x \geq 1 + x$.
\pause
\item Dla dowolnego $n\in\mathbb{N}$ istnieje takie $x_n$, że dla
$x > x_n$ zachodzi $e^x > x^n$.
\end{itemize}
\end{frame}


\section{Wielomiany}

\begin{frame}[fragile]
\frametitle{Polecenie \texttt{pause} -- działania na wielomianach}
\setbeamercovered{transparent}
\begin{enumerate}
\item Suma i różnica wielomianów niezerowych jest albo wielomianem zerowym, albo ma stopień nie większy niż największy ze stopni wielomianów, na których wykonujemy działanie.
\pause
\item  Suma i różnica wielomianów zerowych jest wielomianem zerowym.
\pause
\item Wielomian będący sumą lub różnicą dwóch wielomianów, z których tylko jeden jest niezerowy, ma ten sam stopień co rozważany wielomian niezerowy.
\pause
\item Stopień iloczynu dwóch wielomianów niezerowych jest sumą stopni wielomianów występujących w iloczynie.
\pause
\item Iloczyn wielomianu zerowego przez dowolny wielomian jest wielomianem zerowym.
\end{enumerate}
\end{frame}


\section{Różności}


\begin{frame}
\frametitle{Bloki, definicje, twierdzenia}
\begin{block}{Blok}
Każdy wielomian można zapisać w postaci uporządkowanej.
\end{block}

\begin{definicja}
Jednomianem zmiennej $x$ nazywamy każde wyrażenie algebraiczne, które jest liczbą rzeczywistą lub jest postaci $ax^n$, gdzie $a$ jest liczbą rzeczywistą różną od $0$, a $n$ liczbą całkowitą dodatnią. 
\end{definicja}

\begin{twierdzenie}
Niech $W(x)=a_nx^n+a_{n-1}x^{n-1}+\ldots+a_1x+a_0$ będzie wielomianem o współczynnikach całkowitych takim, że $a_n\neq 0$, $a_0\neq 0$ i $n\geq 1$. Jeśli $W(x)$ ma pierwiastek wymierny $x_0$, przy czym $x_0=\frac{p}{q}$, $p$ jest liczbą całkowitą, $q$ jest liczbą całkowitą dodatnią i $NWD(p,q)=1$, to $p$ jest dzielnikiem liczby $a_0$, a $q$ jest dzielnikiem liczby $a_n$. 
\end{twierdzenie}
\end{frame}


\begin{frame}
\frametitle{Przykłady, alerty}
\begin{przyklad}
\begin{itemize}
\item Wartością wielomianu $W(x) = 6x^3+2x^2-x+1$ dla $x=3$ jest $W(3) = 6\cdot 3^3+2\cdot 3^2-3+1 = 178$.
\item Wartością wielomianu $P(x) = 3x(x+2)(x^2-1)$ dla $x=1$ jest $P(1) = 3\cdot 1\cdot (1+2)(1^2-1)=0$.
\end{itemize}
\end{przyklad}

\begin{alertblock}{To jest tzw. alertblock}
Wśród wielomianów zmiennej $x$ wyróżnia się pewne klasy wielomianów. Jedną z nich jest klasa wielomianów o współczynnikach całkowitych.
\end{alertblock}
\end{frame}



\section{Warstwy}



\begin{frame}
\frametitle{Warstwy}
  \begin{itemize}
  \item<1-> Pierwsza warstwa.
  \item<2-> Druga warstwa.
  \item<3-> Trzecia warstwa.
  \item<4-> Czwarta warstwa.
  \item<5-> Piąta warstwa.
  \end{itemize}
\end{frame}





\begin{frame}
\frametitle{Warstwy cd.}
\begin{definicja}[Jest na trzech warstwach]
Jednomianem zmiennej $x$ nazywamy każde wyrażenie algebraiczne, które jest liczbą rzeczywistą lub jest postaci $ax^n$, gdzie $a$ jest liczbą rzeczywistą różną od $0$, a $n$ liczbą całkowitą dodatnią. 
\end{definicja}
\only<2>{
\begin{definicja}[Tylko na drugiej warstwie]
Jednomianem zmiennej $x$ nazywamy każde wyrażenie algebraiczne, które jest liczbą rzeczywistą lub jest postaci $ax^n$, gdzie $a$ jest liczbą rzeczywistą różną od $0$, a $n$ liczbą całkowitą dodatnią. 
\end{definicja}}
\only<3>{
\begin{twierdzenie}[Tylko na trzeciej warstwie]
Niech $W(x)=a_nx^n+a_{n-1}x^{n-1}+\ldots+a_1x+a_0$ będzie wielomianem o współczynnikach całkowitych takim, że $a_n\neq 0$, $a_0\neq 0$ i $n\geq 1$. Jeśli $W(x)$ ma pierwiastek wymierny $x_0$, przy czym $x_0=\frac{p}{q}$, $p$ jest liczbą całkowitą, $q$ jest liczbą całkowitą dodatnią i $NWD(p,q)=1$, to $p$ jest dzielnikiem liczby $a_0$, a $q$ jest dzielnikiem liczby $a_n$. 
\end{twierdzenie}}
\end{frame}


\section{Kolumny}


\begin{frame}
\frametitle{Kolumny}
\begin{columns}
\column{0.5\textwidth}
\begin{twierdzenie}
Reszta z dzielenia wielomianu $W(x)$ przez dwumian $x-c$ jest równa wartości wielomianu $W(x)$ dla $x=c$, tzn. liczbie $W(c)$.
\end{twierdzenie}
\column{0.5\textwidth}
\begin{twierdzenie}
Jeżeli w dzieleniu wielomianu $W(x)$ przez dwumian $x-c$ reszta z dzielenia jest równa $0$, to wielomian $W(x)$ jest podzielny przez dwumian $x-c$.
\end{twierdzenie}
\end{columns}
\end{frame}









\begin{frame}
  \frametitle{Warstwy -- twierdzenia bloki, itp.}
  \begin{twierdzenie}<1->
    Zbiór liczba naturalnych jest przeliczalny.
  \end{twierdzenie}
  \begin{proof}<3->
    Trzeba poczytać.
  \end{proof}
  \begin{przyklad}<2->
    Zbiór liczb wymiernych jest przeliczalny.
  \end{przyklad}
\end{frame}







\end{document}
