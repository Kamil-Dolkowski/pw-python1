\documentclass[a4paper]{article}


\usepackage[utf8]{inputenc}
\usepackage{polski}

% rodzaje pudelek
% \mbox{tekst} make box 			
% \makebox[szer][poz]{tekst}
% \fbox{tekst} frame box			
% \framebox[szer][poz]{tekst} poz = l,r,s
\begin{document}
Ala ma czternaście kotów w pudełku wygląda następująco

\mbox{Ala ma czternaście kotów}

lub

\fbox{Ala ma czternaście kotów}

lub

\makebox[8cm][r]{Ala ma czternaście kotów}

lub

\makebox[8cm][l]{Ala ma czternaście kotów}

lub

\makebox[8cm][c]{Ala ma czternaście kotów}

lub

\framebox[8cm][r]{Ala ma czternaście kotów}

lub

\framebox[8cm][c]{Ala ma czternaście kotów}

lub

\framebox[8cm][l]{Ala ma czternaście kotów}

\framebox[2mm]{Ala ma czternaście kotów}


Zadanie: zrobić S przekreślone

\makebox[0pt][s]{/}S
% depth - glebokosc
% \totalheight - 

lub jeszcze z totalheight

\makebox[6\totalheight]{Teksty}

\makebox[4\totalheight]{Teksty}

\makebox[2\totalheight]{Teksty}

\makebox[8\totalheight]{Teksty}

\framebox[8\totalheight]{Teksty}

\framebox[6\totalheight]{Teksty}

\framebox[4\totalheight]{Teksty}

\framebox[2\totalheight]{Teksty}
% \raisebox{parametr}{tekst} - podnieść pudełko
% miara wysokosci litery x - 1ex - podwyzszy tekst w lini 
Na przykład opuszczony C\raisebox{-0.7ex}{Z}AC\raisebox{-0.7ex}{Z}A i 
podniesiony \raisebox{1ex}{Czarek}\\
% poz = b - bottom, t - top
\parbox[t]{5cm}{To jest nasz paragraf, który ma 
mieć około 7 cm szerokości To jest nasz paragraf, który ma 
mieć około 7 cm szerokościTo jest nasz paragraf, który ma 
mieć około 7 cm szerokościTo jest nasz paragraf, który ma 
mieć około 7 cm szerokościTo jest nasz paragraf, który ma 
mieć około 7 cm szerokości}
%% przypisy nie w tekscie
\begin{minipage}[t]{4cm}
mieć około 7 cm szerokościTo jest nasz paragraf, który ma 
mieć około 7 cm szerokościTo jest nasz paragraf, który ma 
mieć około 7 cm szerokościTo jest nasz paragraf, który ma 
mieć około 7 cm szerokości
\end{minipage}
\vspace{1cm}

\parbox{4cm}{TO JEST PUDEŁKO O SZEROKOŚCI PEWNEJ 
nie mającej\footnotemark długości}


\end{document}
