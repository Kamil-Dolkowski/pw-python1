\documentclass[a4paper]{article}

\usepackage{latexsym}
\usepackage{amsmath}
\usepackage{amscd}
\usepackage{amssymb}

\usepackage[utf8]{inputenc}
\usepackage{polski}
\usepackage[margin=2cm]{geometry}

%\usepackage{qtimes}
%\usepackage{qpalatin}
%\usepackage{qswiss}
%\usepackage{qbookman}
%\usepackage{qzapfcha}


%%%% kurier
%\usepackage{kurier}
%\usepackage[light]{kurier}
%\usepackage[condensed]{kurier}
%\usepackage[light,condensed]{kurier}
%\usepackage[math]{kurier}
%\mathversion{kurier}
%\mathversion{kurierbold}


%%% antykwa toruńska
%\usepackage{anttor}
%\usepackage[QX]{fontenc}
%\usepackage[light]{anttor}
%\usepackage[condensed]{anttor}
%\usepackage[light,condensed]{anttor}
%\usepackage[math]{anttor}
%\mathversion{anttor}
%\mathversion{anttorbold}


%%% iwona
%\usepackage{iwona}
%\usepackage[QX]{fontenc}
%\usepackage[light]{iwona}
%\usepackage[condensed]{iwona}
%\usepackage[light,condensed]{iwona}
%\usepackage[math]{iwona}
%\mathversion{iwona}
%\mathversion{iwonabold}


\begin{document}
Wielomiany ortogonalne stanowią ważną klasę układów ortogonalnych\footnote{To jest przypis}, które mają duże znaczenie w analizie, głównie dzięki
{\slshape możliwości} $\cos x$ \textsl{rozwijania} dowolnych funkcji $\arcsin x$ należących do bardzo obszernych klas funkcji w szeregi według funkcji ortogonalnych.
Przykładami takich szeregów mogą być szeregi Fouriera, szeregi Fouriera-Bessela itd.
\[
\cos (\alpha + \beta) = \cos \alpha \cos\beta - \sin \alpha \sin \beta
\]
Wielomiany Czebyszewa znane też były wcześniej w 1748 roku Leonhardowi Eulerowi w związku z rozkładem funkcji 
$\cos nx$ według potęg $\cos x$. Badaniem wielomianów ortogonalnych zajmował się również niemiecki matematyk Carl Gustaw 
Jacobi, który wprowadził wielomiany nazwane jego nazwiskiem tzw. wielomiany Jacobiego, uogólniające wielomiany Legendre'a.

\begin{enumerate}
\item Metody porównawcze stosuje się w wielu dziedzinach geometrii. W geometrii riemannowskiej
\item są one kojarzone z podstawowymi twierdzeniami Raucha, Toponogowa. Ideę przeniesienia 
\item ubiegłego stulecia. Ich rozwój jest związany z wynikami Gromowowa z lat 70-tych i 80-tych.
\end{enumerate}

\end{document}
